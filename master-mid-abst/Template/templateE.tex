\documentclass[twocolumn, a4paper]{UECIEresumeE}

\usepackage[dvipdfmx]{graphicx}
\usepackage{graphicx}
\usepackage{amsmath}
\usepackage{txfonts}

\title{Format for Graduation/Master Thesis Resume}
\date{MM, DD, YYYY}
\affiliation{Dept. of Informatics, Faculty of Information and Engineering, UEC.}
\supervisor{Prof. YYY, and Asoc. Prof. ZZZ}
\studentid{000000}
\author{Foo Bar}
\headtitle{Graduation Thesis presentation (Interim) YYYY}
%\headtitle{Graduation Thesis presentation YYYY}
%\headtitle{Master Thesis presentation (Interim) YYYY}
%\headtitle{Master Thesis presentation YYYY}

\begin{document}
\maketitle

\section{Introduction}
From the viewpoint of the audience, this section helps understanding for your motivation.
Thus, you should NOT write your work at this first section.
Introduction requires the background for your research theme, 
which makes clear your standing point.
Citations might be important for explaining the previous work.
After that, you should clear the problem in this research, and the overview of the pathway to solve it.


\section{Format}
In the next section, you should start to write your method, model and so on for your work.
Figure and tables are effective for explaining.

Moreover, you should use paragraph properly. 
A paragraph, which means an unit of the logic in the text, helps the understanding of the audience for your work. Without proper paragraphing, your text might be hard to understanding.

Detail format for the resume is followings:
\begin{itemize}
  \item{Paper size: A4, and 2 pages}
  \item{Two column}
  \item{Top and side margins are 25mm}
  \item{The first page composed from title, presenter (course name, your id, your name), and your supervisor}
  \item{You should not also run over figures or tables above the margins}
  \item{Title should be presented in Gothic 12pt, and the body should Roman 9--10 pt}
  \item{You should set the header string properly}
  \item{The section name should be emphasize by use of Gothic font.}
\end{itemize}
If you use \LaTeX, it is easy to make by use of the including class file.

Additionally, you should use mathematical mode, if you express the formula and so on.
Using MS-word, you should use such editor like 'Math-type' and so on.


\section{Representation of the result}
You should use figures and tables for easy understanding.
In this proceedings, the page length is limited only 2 pages, so that,
you should use a few conclusive figures of tables.
When you use figures or tables, you must explain in the main body as well as its captions.


\section{Conclusion}
In the last section, it is better to summarize your work. If you have several future works, you could write them here, however, you should remember that too much future works could backfire.

In the end of the resume, you should show the citation list.
The citation is a respect for the previous work, so that it also be a 
important part\cite{Kinoshita1}.


{\small
\begin{thebibliography}{*}
\bibitem{Kinoshita1} Koreo Kinoshita, \textit{Rikakei no Sakubun Gijyutu (in Japanese)}, Chuko-Shinsho 624, 1981.
\end{thebibliography}
}
\end{document}